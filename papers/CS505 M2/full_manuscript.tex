\documentclass[12pt]{article}

% Package imports
\usepackage{amsmath}    % Math symbols and equations
\usepackage{graphicx}   % Images
\usepackage{url}        % Formatting URLs
\usepackage{hyperref}   % Clickable links in the PDF
\usepackage{fancyhdr}   % Custom headers and footers
\usepackage{geometry}   % For page layout and margins
\geometry{margin=1in}   % 1-inch margins all around
\usepackage{times}      % Times font, common for academic papers
\usepackage{caption}    % Captioning for figures and tables
\usepackage{float}      % Enhanced control over figure and table positioning

% Header and Footer
\pagestyle{fancy}
\fancyhf{}
\fancyhead[L]{Your Name}
\fancyhead[R]{Paper Title}
\fancyfoot[C]{\thepage}

% Title Information
\title{Bb}
\author{Chris Ziko, Rakin Munim \\ Department of Computer Science \\ Boston University}
\date{\today}

\begin{document}

% Title Page
\maketitle

% Abstract
\begin{abstract}
This is a final paper on research done in CS505. This paper is a repository for ideas, links, and more. 
\end{abstract}

% Keywords
\textbf{Keywords:} music, deep-learning, language-modeling

% Introduction
\section{Introduction}
The introduction provides background on the topic, states the motivation for the work, and briefly describes the structure of the paper.

% Related Work
\section{Related Work}
Discuss previous studies and frameworks related to your topic, including citations. For example, you could refer to a study \cite{raffel2016lakhmidi}.

% Methodology
\section{Methodology}
Describe the methodology used in your research, including any specific techniques or tools applied.

\section{Data}

\subsection{Definitions}
\textbf{degeneration} : This is an important word when it comes to cognitive science. Degenerative works, texts, etc are ones that look normal but make no sense. This is important because we need to understand how degenerative any model is. Example sentence: "ChatGPT is successful because its not degenerative when it comes to high school level prompts." 



\subsection{Data}
The data set we used for our project is the Lakh dataset. The Lakh dataset is collection of midi files that have been aligned with 

\subsection{Data Analysis}
Explain the techniques or algorithms used to analyze the data.

% Results
\section{Results}
Present the findings of your study. Figures and tables are helpful here.

% Sample Figure


% Discussion
\section{Discussion}
Provide insights, interpretations, and implications of the results here.

% Conclusion
\section{Conclusion}
Summarize the main points, highlight the contributions, and suggest directions for future work.

% Acknowledgments (Optional)
\section*{Acknowledgments}
You can include acknowledgments here.

% References
\bibliographystyle{IEEEtran}
\bibliography{anthology,custom}


\end{document}
